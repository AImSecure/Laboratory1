% abstract.tex

\begin{abstract}
This report documents the development of a supervised intrusion-detection pipeline for the CICIDS2017 dataset using feed-forward neural networks. We first align the data with standard machine-learning practice by removing corrupted or duplicated flows, stratifying train/validation/test splits, and applying feature scaling tailored to the observed outlier distribution. Building on this foundation, we benchmark single-hidden-layer models, study the inductive bias introduced by the destination port feature, and progressively extend the architecture depth while varying batch size, optimizer, and loss formulations.

\noindent Our experiments show that ReLU-activated shallow models already capture most benign and frequent attack patterns, while class-weighted cross-entropy substantially improves the recall of rare ports and brute-force events. Deeper configurations offer additional gains in macro-averaged performance and training stability, provided that mini-batch sizing and optimization hyperparameters are co-tuned. Finally, regularization strategies based on dropout, batch normalization, and weight decay mitigate overfitting when the network capacity increases, yielding models that remain robust once high-leverage features are neutralised. These results highlight the importance of carefully balancing architecture complexity and regularization to deploy resilient FFNN-based intrusion detectors.
    
\end{abstract}
