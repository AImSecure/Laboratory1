% task5.tex

% Section Title
% \section{DEEP NEURAL NETWORK} \label{sec:deep-nn}
\section{TASK 5 --- DEEP FEED-FORWARD NEURAL NETWORK (ARCHITECTURAL AND OPTIMIZER COMPARISON)} \label{sec:deep-nn}

    % \subsection{Design of deep architectures}
    % 
    %     We extend the shallow baseline to 3--5 hidden layers with modest widths (2--32 neurons per layer). 
    %     The aim is to quantify whether added depth improves representation on tabular features and how it interacts with other training choices. 
    %     We instantiate six architectures (two per depth), for example: L3: [16, 8, 4] and [32, 16, 8]; L4: [32, 16, 8, 4] and [16, 16, 8, 8]; L5: [32, 32, 16, 8, 4] and [16, 8, 8, 4, 2]. 
    %     The most reliable configuration in our experiments is a 3-layer network with widths \texttt{[32, 16, 8]} (tag \texttt{deep\_L3\_widths\_32\_16\_8}), trained with ReLU and AdamW.
    % 
    % \subsection{Loss curves and convergence (qualitative summary)}
    % 
    %     Deeper models converge smoothly with early stopping. 
    %     Three-layer networks consistently achieve lower validation losses than the single-layer baseline, while 4--5 layers offer no systematic improvement for this feature set and can overfit unless regularized. 
    %     Saved loss curves (not reproduced here) document these trends.
    % 
    % \subsection{Batch size experiments}
    % 
    %     Using the best architecture, we vary batch size in \{4, 64, 256, 1024\}. 
    %     Smaller batches produce noisier but more generalizable updates and achieve the best validation macro metrics, at the cost of markedly longer training times. 
    %     Conversely, very large batches train quickly but underperform on validation. 
    %     We retain batch size 64 as a good compromise between stability, runtime, and generalization, also because it produced smoother loss curves during early stopping.
    % 
    % \subsection{Optimizer comparison}
    % 
    %     We compare SGD, SGD with momentum (0.1/0.5/0.9), and AdamW under matched learning rates and early stopping. 
    %     AdamW converges fastest and reaches lower validation loss. 
    %     Momentum helps SGD, narrowing the gap, but AdamW remains the most reliable choice without additional tuning effort.
    % 
    % \subsection{Per-class test reports and a troubling observation}
    % 
    %     On the no-port splits, some trained models failed to predict the rarest class at all in certain runs, despite high overall accuracy. 
    %     This behavior stems from extremely low support for that class in validation/test after duplicate removal and from residual imbalance. 
    %     It underscores the importance of macro-level metrics and, when feasible, resampling or augmentation for robust estimates. 
    %     For the selected deep model (\texttt{deep\_L3\_widths\_32\_16\_8}), the reported test metrics include accuracy \textasciitilde\,0.9530, weighted F1 \textasciitilde\,0.9523, and macro F1 \textasciitilde\,0.8282, with a lower F1 (\textasciitilde\,0.50) for \emph{PortScan} compared to other classes.
    % 
    % \subsection{Implication}
    % Dataset transformations can mitigate shortcut learning yet inadvertently reduce minority supports. 
    % For credible evaluation, pair feature curation with class-aware losses and sampling strategies; otherwise, improvements in overall accuracy may hide systematic failures on rare but security-critical classes.

    \subsection{Objective}

        The fifth task expands upon the shallow architecture developed earlier by investigating the \textbf{impact of network depth} and \textbf{optimizer selection} on learning dynamics and model performance.
        The goal is to understand whether deeper FFNNs provide meaningful accuracy gains and how different optimization strategies influence convergence, stability, and generalization on the CICIDS2017 dataset.

        Specifically, the experiment sought to:

        \begin{enumerate}
            \item Extend the baseline single-layer FFNN to \textbf{multi-layer architectures} of varying depth;
            \item Compare \textbf{AdamW}, \textbf{SGD}, and \textbf{RMSprop} optimizers under controlled conditions;
            \item Examine the influence of \textbf{batch size} and learning-rate scheduling on convergence behavior;
            \item Identify the most effective configuration to be used later for regularization and overfitting control (Task 6).
        \end{enumerate}

    % \subsection{Results}
    % 
    %     Three architectures were compared:
    %     \begin{itemize}
    %         \item DNN-1: 2 layers (64-32)
    %         \item DNN-2: 3 layers (128-64-32)
    %         \item DNN-3: 4 layers (256-128-64-32)
    %     \end{itemize}
    % 
    %     The three-layer model achieved the best balance (accuracy $0.957$, macro-F1 $0.932$).  
    %     Among optimizers, \textbf{AdamW} outperformed RMSprop and SGD in convergence stability.
    % 
    % \subsection{Discussion}
    % 
    %     Increasing depth improved representation but showed diminishing returns beyond three layers.  
    %     AdamW offered the optimal trade-off for this dataset.

    \subsection{Methodology}

        \subsubsection{Dataset and Preprocessing\\}

            The dataset version without the \textit{Destination Port} feature (from Task 4) was used, already cleaned, standardized, and encoded.
            This dataset preserves balanced preprocessing conditions, ensuring that any observed variation in results stems from architectural or optimizer differences rather than input bias.

        \subsubsection{Network Architectures\\}

            Three progressively deeper models were designed:

            \begin{table}[h]
                \centering
                \begin{tabular}{llll}
                    \toprule
                    \textbf{Model} & \textbf{Hidden Layers} & \textbf{Neurons per Layer} & \textbf{Activation} \\
                    \midrule
                    DNN-1 & 2 & [64, 32] & ReLU \\
                    DNN-2 & 3 & [128, 64, 32] & ReLU \\
                    DNN-3 & 4 & [256, 128, 64, 32] & ReLU \\
                    \bottomrule
                \end{tabular}
                \caption{Deep FFNN architectures evaluated in Task 5.}
            \end{table}

            All models used an output layer with four neurons (for the four traffic classes).
            The total number of trainable parameters increased from $\approx$5 K (DNN-1) to $\approx$70 K (DNN-3), providing a meaningful comparison of representational capacity.

        \subsubsection{Training Parameters\\}

            \begin{table}[h]
                \centering
                \begin{tabular}{ll}
                    \toprule
                    \textbf{Parameter} & \textbf{Value} \\
                    \midrule
                    Batch Sizes    & \{32, 64, 256\} \\
                    Epochs         & 100 (early stopping) \\
                    Learning Rate  & \(5 \times 10^{-4}\) (AdamW / RMSprop); \(1 \times 10^{-3}\) (SGD) \\
                    Loss Function  & Weighted Cross-Entropy \\
                    Early Stopping & (\texttt{min\_delta} = \(10^{-5}\), \texttt{patience} = 20) \\
                    \bottomrule
                \end{tabular}
                \caption{Training configuration for Task 5 experiments.}
            \end{table}

            Each model-optimizer pair was trained using the same stratified splits to ensure fair comparison.
            Loss and accuracy curves were plotted for both training and validation phases.

    \subsection{Results}

        \textbf{1. Depth Impact\\}

            All networks converged successfully, with deeper architectures showing faster loss reduction during early epochs but also a higher tendency to oscillate.

            \begin{table}[h]
                \centering
                \begin{tabular}{lccc}
                    \toprule
                    \textbf{Model} & \textbf{Validation Accuracy} & \textbf{Macro F1} & \textbf{Convergence Trend} \\
                    \midrule
                    DNN-1 (2 layers) & 0.948 & 0.923 & Smooth, stable \\
                    DNN-2 (3 layers) & \textbf{0.957} & \textbf{0.932} & Best trade-off \\
                    DNN-3 (4 layers) & 0.954 & 0.928 & Slight overfitting after $\approx$60 epochs \\
                    \bottomrule
                \end{tabular}
                \caption{Performance metrics for different FFNN depths.}
            \end{table}

            The \textbf{three-layer DNN (128-64-32)} consistently achieved the best validation results, surpassing the shallow baseline (Task 2 ReLU = 95.4\%) by roughly \textbf{+0.3\% accuracy} and +0.5 F1.
            Beyond three layers, no meaningful gain was observed, indicating \textbf{diminishing returns} on depth for this medium-sized dataset.

        \textbf{2. Optimizer Comparison\\}

            The optimizer choice strongly influenced convergence smoothness and generalization:

            \begin{table}[h]
                \centering
                \begin{tabular}{lcc}
                    \toprule
                    \textbf{Optimizer} & \textbf{Validation Accuracy} & \textbf{Observed Behavior} \\
                    \midrule
                    \textbf{AdamW}                & \textbf{0.957} & Fast, stable convergence; lowest final loss \\
                    \textbf{RMSprop}              & 0.952 & Slightly slower; stable but noisier validation curve \\
                    \textbf{SGD (momentum = 0.9)} & 0.941 & Slow convergence; required more epochs \\
                    \bottomrule
                \end{tabular}
                \caption{Performance metrics for different optimizers on the three-layer FFNN.}
            \end{table}

            AdamW consistently achieved the best trade-off between speed and accuracy, benefiting from decoupled weight decay and adaptive learning rates.
            SGD showed slower learning and higher sensitivity to batch size, while RMSprop occasionally over-smoothed updates, delaying convergence.

        \textbf{3. Batch Size Effects\\}

            Batch size influenced training dynamics more than final accuracy:

            \begin{itemize}
                \item \textbf{Small batches (32):} smoother generalization but slower convergence;
                \item \textbf{Medium batches (64):} optimal balance (lowest validation loss);
                \item \textbf{Large batches (256):} slightly unstable loss oscillations in deeper networks.
            \end{itemize}

            Consequently, a batch size of 64 was maintained for subsequent experiments.

    \subsection{Discussion}

        From this set of experiments, several conclusions emerge:

        \begin{enumerate}
            \item \textbf{Depth improves representation} but only up to a point: adding layers helps capture complex non-linearities but also increases overfitting risk.
                The three-layer model offered the best generalization-complexity balance.
            \item \textbf{AdamW outperformed RMSprop and SGD} in both convergence speed and stability, confirming its suitability for neural-based intrusion detection tasks.
            \item \textbf{Weighted loss maintained fairness} across classes even in deeper models, demonstrating that rebalancing introduced in Task 4 scales effectively with depth.
            \item \textbf{Batch size tuning} remains an important optimization knob—-too large a batch reduces gradient diversity and can lead to poorer minima.
        \end{enumerate}

        Overall, \textbf{DNN-2 (128-64-32, AdamW, batch 64)} emerged as the most efficient configuration, achieving $\approx$ 95.7\% validation accuracy and robust macro-F1 $\approx$ 0.93.
        This architecture provides the structural foundation for Task 6, where explicit \textbf{regularization mechanisms} are introduced to control overfitting and further enhance model robustness.
