% introduction.tex

% Section Title
\section{INTRODUCTION} \label{sec:introduction}

This laboratory explores the implementation of a \textbf{Feed-Forward Neural Network (FFNN)} using the \textbf{CICIDS2017} dataset, a standard benchmark for intrusion detection research.  
    The goal is to construct a complete machine learning pipeline in \cooltext{PyTorch}—from raw data preparation to model evaluation—to analyze how architectural choices and preprocessing strategies affect classification performance in cybersecurity contexts.

    \noindent The experiment is divided into six tasks, progressively building complexity:
    \textbf{data preprocessing}: cleaning, scaling, and outlier management;
    \textbf{baseline FFNN training}: single hidden layer;
    \textbf{feature bias analysis}: focus on \textit{Destination Port};
    \textbf{loss-function weighting}: for class imbalance;
    \textbf{deep network optimization}: architecture and optimizer variations;
    \textbf{regularization}: dropout, batch normalization, weight decay.

\noindent The overarching aim is to understand how data characteristics and network configuration influence the model's ability to detect attack types such as \textit{DoS Hulk}, \textit{PortScan}, and \textit{Brute Force}, while maintaining robustness and generalization.